    \documentclass[12pt]{article}

% If you just want to change part of the text, use
%{\fontsize{11pt}{12pt}\selectfont text here}


\usepackage{graphicx}
\DeclareGraphicsExtensions{.eps,.ps,.tif,.gif,.bmp,.jpg}

\newcommand{\mycaption}[2]{\parbox{\linewidth}{\fontsize{9pt}{9pt}\selectfont{\bf Figure #1. } #2}}

\newcommand{\mycaptext}[2]{\parbox{\linewidth}{\fontsize{9pt}{9pt}\selectfont{\bf #1 } #2}}

\include{defont2}
\usepackage{fullpage}
\usepackage{amsmath}
\linespread{0.9}
%\linespread{1.5}



\makeatletter
\renewcommand \thefigure
     {\@arabic\c@figure}
\long\def\@makecaption#1#2{%
  \vskip\abovecaptionskip
  \sbox\@tempboxa{\fontsize{9pt}{10pt}\selectfont{\bf#1:} #2}%
  \ifdim \wd\@tempboxa >\hsize
    {\fontsize{9pt}{10pt}\selectfont{\bf #1:} #2}\par
  \else
    \global \@minipagefalse
    \hb@xt@\hsize{\hfil\box\@tempboxa\hfil}%
  \fi
  \vskip\belowcaptionskip}
\makeatother


%\setlength{\parskip}{10pt}
\setlength{\parindent}{0pt}
\addtolength{\leftmargin}{2mm}
%\setlength{\parindent}{2mm}
%\setlength{\textheight}{224mm}
%\setlength{\topmargin}{-2mm}
%\setlength{\textwidth}{156mm}
%\setlength{\oddsidemargin}{3mm}
%\addtolength{\textheight}{8mm} 
%\addtolength{\topmargin}{-4mm}
\setlength{\itemsep}{0.1em}

 \begin{document}

 NEW COURSE FOR SPRING 2011 !!!
 
 {\ittv   Undergraduate Research Seminar:}\nl
{\fontsize{14pt}{14pt}\selectfont {\bftv Remote Sensing, Signal Processing and Spatial Data Analysis\nl
(ECEN 4004, M/W 3:30-4:45  in ECCR 137, 3 credit hours)\nl
%{\bf Spatial Statistical Analysis and Signal Processing
% of Airborne and Satellite Remote-Sensing Data ---
% Engineering Perspectives and Science Applications (ECEN 4004, M/W 4:00-5:30) \nl
Faculty: Dr. Ute Herzfeld, Associate Research Professor, ECEE }}

\ss
{\bftv Explore new topics like a research team:}\nl
-- remote sensing of the environment\nl
-- Earth observation technology\nl
-- passive and active microwave\nl
-- spatial data analysis\nl
-- signal processing\nl
-- image analysis, interpolation and extrapolation\nl
-- new technology for observations from satellites, manned and unmanned aircraft\nl
-- instrument and survey design\nl
-- fascinating applications: Greenland and Antarctica, ice sheets, oceans, climate change, \nl
 \~~~~~ forests,
      ecology, exploration and mining,......

{\bftv Things to know:}\nl
--  course  style will be centered on research, problem-solving and skill development\nl
-- you need not have background in remote sensing or spatial statistics.\nl
-- course counts as technical or general elective\nl
-- learn skills that prepare for a job at CU and in the professional world

% For example: Which data situation would require interpolation? Which mathematical tool can I use to 
% solve this problem?
%How do I find existing solutions, or can I design my own? 

{\bftv WHO?  UNDERGRADUATES}\nl 
-- in: ECEE, AES,...all engineering majors, (applied) mathematics, computer sciences, physics,
 \~~~ physical sciences including PAOS, geology, geography, environmental sciences, glaciology, 
  \~~~ geophysics, ecology and hydrology\nl
-- enthusiasm and interest are more important than specific course background\nl
-- you need: about 15 credit hours towards your degree (such as basic engineering, maths,  \~~~ computer fitness, and/or science)
%-- contact Ute.Herzfeld@colorado.edu (questions? prerequisites?)\nl

{\bftv Assessments:}\nl
 homeworks and a  special project on a topic of your choice (1 or 2 students per project)

{\bftv more info?} http://cires.colorado.edu/~herzfeld/utecourses.html\\
%http://ecee.colorado.edu/fac\_staff/personnel\_pages/herzfeld.html
-- contact Ute.Herzfeld@colorado.edu (questions? prerequisites?)

{\bftv Instructor:}  Ute Herzfeld is an Associate Research Professor at ECEE, a Senior Research Scientist  at the Cooperative Institute for Research in Environmental Sciences (CIRES) and an Affiliated Professor  in Applied Mathematics. Current research interests include
mathematical concepts in geophysics, cryospheric sciences,  ecology, oceanography and remote sensing,
spatial complexity,  and development of instrumentation for spatial Earth observation (ICESat-2, CryoSat, and UASs).

\end{document}







